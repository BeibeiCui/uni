\chapter{Quadratic Assignment Problem}\label{appendixA}
Consider a problem of assignment of $n$ facilities to $n$ locations given the transportation costs between the locations depending on the flow between them and opening costs of facilities in certain locations. The aim is to minimize the summary cost of the assignment. Let $D=(d_{kl}),F=(f_{kl}), B=(b_{ik})\in\mathbb{R}^{n\times n}$ be a real matrices that define a distances ans flow between the locations, as well as opening costs. The problem defined above can then be formulated as an integer quadratic program~\cite{Burkard98thequadratic,Koopman_Backman}:
\begin{equation}\label{eq:QAP_classic}
P = \argmin_{\hat{\sigma}\in\Sigma_n}\sum_{i=1}^{n}\sum_{j=1}^{n}f_{ij}d_{\hat{\sigma}(i)\hat{\sigma}(j)}+\sum_{i=1}^{n}b_{i\hat{\sigma}(i)},
\end{equation}
where $\Sigma_n$ is a set of all possible permutations of the set $\{1,\dots,n\}$, and is called \emph{Koopmans-Beckmann version of the quadratic assignment problem} (further \emph{QAP}). %We assume further, that the matrix $B$ is a constant matrix. The second term in the objective function of~\eqref{eq:QAP_classic} can be left without 

%Here we want to show, how the formulation~\eqref{eq:QAP_classic} can be transformed into \eqref{eq:QAP1} and \eqref{eq:QAP2}. 

We can assign a permutation matrix $P=(P_{ij})\in\{0,1\}^{n\times n}$ to each permutation $\sigma$, where $P_{i\sigma(i)}=1$ and $0$ elsewhere. The set of all feasible permutation matrices is defined as
\begin{equation*}
\Pi_n=\{P\in\{0,1\}^{n\times n}|\sum_{i=1}^{n}P_{ij}=\sum_{j=1}^{n}P_{ij}=1\quad\forall i,j=1,\dots,n\}.
\end{equation*}
It is easy to see that, the formulation~\eqref{eq:QAP_classic} %with zero-matrix $B$
is equivalent to
\begin{equation}\label{eq:QAP_classic2}
P = \argmin_{\hat{P}\in\Pi_n}(F\cdot \hat{P}D\hat{P}^T)+B\cdot\hat{P},
\end{equation}
where $\cdot$ denotes the Frobenius inner product of two square matrices defined as $A\cdot B=\sum_{i=1}^{n}\sum_{j=1}^{n}A_{ij}B_{ij}$.

We recall shortly the definition of the Kronecker product of two matrices and it's connection with the Frobenius inner product of two matrices and matrix trace.

The Kronecker  product of two matrices $A=\{A_{ij}\},B=\{B_{ij}\}\in\mathbb{R}^{n,n}$ is a new $n^2\times n^2$ matrix C
\begin{equation}\label{eq:kronecker}
C=A\otimes B =  	
\begin{pmatrix} a_{11}B & \dots & a_{n1}B \\ \vdots & \ddots & \vdots \\ a_{1n}B & \dots & a_{nn}B  \end{pmatrix}.
\end{equation}

From the definition of the matrix trace follows~\ToDo{cite}:
\begin{equation}\label{eq:Property1}
A\cdot B=\sum_{i=1}^{n}\sum_{j=1}^{n}A_{ij}B_{ij}=\sum_{i=1}^{n}(AB^T)_{ii}=\tr(AB^T)=\vec(A)^T\vec(B),
\end{equation}
where $\vec(A)$ denotes the column-wise vectorization of a matrix $A$. Additionally, it holds~\ToDo{cite}:
\begin{equation}\label{eq:Property2}
\vec(APB)=(B^T\otimes A)\vec(P).
\end{equation}

We can now apply the equality $2$ and $3$ in~\eqref{eq:Property1} to the formulation~\eqref{eq:QAP_classic2} and get the~\emph{trace formulation of QAP}~\cite{Burkard98thequadratic}:
\begin{equation}\label{eq:QAP_classic3}
P = \argmin_{\hat{P}\in\Pi_n}\tr(F\hat{P}D^T\hat{P}^T+B\hat{P}^T).
\end{equation}
The objective function of~\eqref{eq:QAP_classic3} can be further rewritten based on the last equality in~\eqref{eq:Property1} and property~\eqref{eq:Property2} as follows:
 \begin{align}\label{eq:QAP_classic4}
\tr(F\hat{P}D^T\hat{P}^T+B\hat{P}^T)&=\tr(\hat{P}(F\hat{P}D^T)^T)+\vec(B)^T\vec(\hat{P}) \notag\\	
									&=\vec(\hat{P})^T\vec(F\hat{P}D^T)+\vec(B)^T\vec(\hat{P}) \notag\\	
                                    &=\vec(\hat{P})^T(D\otimes F)\vec{\hat{P}}+\vec(B)^T\vec(\hat{P}).
\end{align}
The derived formulation of the \emph{QAP} is obviously the same as in~\eqref{eq:QAP2}, where $F=(A^I)^T$, $D=(A^J)^T$ and the matrix $B$ is a constant matrix with same values, which than does not have any influence on the optimization problem. It remains to show, that \eqref{eq:QAP1} is equivalent to \eqref{eq:QAP_classic2}.

Indeed, consider the objective function in~\eqref{eq:QAP1}:
\begin{align*}
\|A^I-\hat{P}A^J\hat{P}^T\|^2 &= \tr((A^I-\hat{P}A^J\hat{P}^T)^T(A^I-\hat{P}A^J\hat{P}^T)) \notag\\
							  &= \tr(((A^I)^T-\hat{P}(A^J)^T\hat{P}^T)(A^I-\hat{P}A^J\hat{P}^T)) \notag\\
							  &= \tr((A^I)^TA^I-(A^I)^T\hat{P}A^J\hat{P}^T-\hat{P}(A^J)^T\hat{P}^TA^I+\hat{P}(A^J)\hat{P}^T\hat{P}A^JP^T) \notag\\
							  &= \tr((A^I)^TA^I-2(A^I)^T\hat{P}A^J\hat{P}^T+A^J(A^J)^T).
\end{align*}
The first and the last term are obviously constant, and thus can be ignored in optimization. Finally, substitution of $F=(A^I)^T$, $D=(A^J)^T$ leads us to the formulation~\eqref{eq:QAP_classic2}. That is exactly, what we wanted to show.



