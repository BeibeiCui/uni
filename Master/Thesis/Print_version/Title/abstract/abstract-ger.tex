Graph Matching ist eines der grundlegenden Probleme in der Graphentheorie und Computer Vision. Aufgrund seiner praktischen Relevanz ist es auch ein ausgiebig erforschtes Problemfeld. Es existieren viele approximative Algorithmen, die in der Lage sind schnell eine hoch qualitative Lösung zu liefern. Allerdings sind viele der Algorithmen nur für kleine Graphen mit bis zu 100 Knoten geeignet und lassen sich schwer für größere Graphen anwenden. Aus diesem Grund haben wir uns in dieser Masterarbeit mit der Entwicklung eines neuen Ansatzes zum Graph Matching beschäftigt, der die Anwendung von existierenden Algorithmen für große Graphen mittels eines zweistufigen Ansatzes ermöglicht. Zwei gegebene Graphen sind auf der unteren Stufe platziert. Um die Schwierigkeit des Problems zu minimieren, zerlegen wir jeden einzelnen Graphen in eine fixe Anzahl von Teilgraphen, die wir mit einem Knoten eines neuen Graphen (Ankergraphen) erfassen. Die zwei Ankergraphen repräsentieren die zweite Stufe unseres Verfahrens. Zuerst finden wir die Zuordnung zwischen den Knotenmengen der beiden Ankergraphen. Um die Abbildung zwischen den Knoten der ursprünglichen Graphen zu finden lösen wir das Matchingproblem  für jedes zugeordnete Paar von Teilgraphen parallel. Die Vorgehensweise wird durch eine Update-Regel erweitert und iterativ wiederholt. Wir demonstrieren die Funktionalit\"{a}t unseres Ansatzes mit Beispielen von künstlich generierten Graphen und der Zuordnung von Merkmalpunkten auf zwei Bildern.