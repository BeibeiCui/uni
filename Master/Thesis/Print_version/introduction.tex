\chapter{Introduction}
Graph theory is one of the oldest and widely used branches of discrete mathematics. Graphs have found an application in almost all fields of computer science, including image processing and computer vision. The reason for such a success is their simple way to model pairwise relationships between different objects. In terms of image processing and computer vision those objects can be presented by image regions, image features or even separate pixels. Such a graph representation often helps to transform an existing practical problem into a good investigated problem of graph theory. An example is image matching, which is one of the central problems in computer vision. %object recognition.
Using a graph representation of images, which for example can be obtained by connecting selected points of images with edges, this problem can be formulated as a graph matching problem. Although the last is not easy to solve (in most cases the graph matching problem is NP-hard), there are a lot of approximative algorithms that solve it in polynomial time.

Further in this work we will see, that one distinguishes two general types of graph matching problems: exact and inexact ones. As the exact graph matching is often too strict to be applied to practical applications, such as image matching, we concentrate ourselves on the problems that formulate the graph matching problem in an inexact way.
In the most general case algorithms for solving inexact matching problems use a so-called affinity or similarity matrix of two graphs, which contains information how similar they are.
However in a naive implementation such algorithms have strong limitations on the size of the graphs they are able to handle in reasonable time due to high time and memory demand. Knowing this issues the aim of this thesis is to present a novel framework that should help to extend the usability of existing graph matching algorithms to bigger graphs. The proposed framework helps to reduce the complexity of the problem by subdividing it into smaller problems, which can be easily handled with existing algorithms. %This technique can be seen as an adoption of the well known divide and conquer paradigm to graph matching.
This technique can be seen as a variant of the well-known divide and conquer paradigm.

This thesis is organized as follows. In chapter~\ref{chapter:GM} we give a general formulation of the graph matching problem together with its possible specifications, which are determined by required properties of the matching. Additionally we show how different formulations are related to each other. We also provide an extensive overview of existing algorithms for solving graph matching problems, which contains classical widely known works as well as recently published researches in this field.
Chapter~\ref{chapter:2levelGM} describes the novel two level graph matching framework based on a variation of the divide-and-conquer paradigm. %We explain in detail each single step of the purposed technique and . 
In chapter~\ref{chapter:results} we report results of the application of the proposed framework to synthetic generated graphs and real images. The last chapter gives a summary of the achieved results. There we also discuss possible improvements of the developed framework and future work.
