\documentclass[11pt,a4paper]{scrartcl}

\usepackage{fullpage}
%\usepackage[top=2in, bottom=1.5in, left=1in, right=1in]{geometry}

\usepackage[ngerman]{babel}
\usepackage[utf8]{inputenc}
\usepackage{graphicx}
\usepackage{rotating}

\usepackage{amsmath}
\usepackage{amsfonts}
\usepackage{amsthm}
\usepackage{mathtools}

\usepackage{listings}
\usepackage{color}
	\definecolor{darkred}{rgb}{0.55, 0.0, 0.0}
	\definecolor{darkgreen}{rgb}{0.19, 0.80, 0.19}
%%%%% COMMANDS

%\renewcommand{\thesubsection}{\alph{subsection})}

\usepackage{listings}
\lstset{language=Python,
basicstyle=\scriptsize,
keywordstyle=\color{blue}\ttfamily,
stringstyle=\color{red}\ttfamily,
commentstyle=\color{darkgreen}\ttfamily,
numbers=left,
title=\lstname
}

\setlength{\parindent}{0pt} 

\begin{document}

\title{Programmieraufgabe 2}
\author{Ekatherina Tikhoncheva}
\maketitle

\section{Der Basis-Decoder}

Das kommentierte Skript {\bf decode}, so wie es gegeben wurde, ist unter den Namen {\bf decodeCommented} gespeichert.
 
Das beste Ergebnis (LM+TM) bei der Stackgröße $1$, das man mit diesem Skript erzielen kann, ist $-1442.727775$.
Wenn man die Stackgröße auf $1000$ erhöht, ist das Ergebnis (LM+TM) besser geworden : $-1439.213923$.

\section{Decoding mit Phrasenvertauschung}

Wie man gesehen hat, behält der Basis Decoder die Reihenfolge der französischen Phrasen bei. Wir möchten das ändern, so dass der neue Decoder benachbarte Phrasen vertauschen kann. 

\vspace{10pt}
Um die Implementierungsidee zu erklären betrachten wir ein kleines Beispiel. Sei der französischen Satz {\it un Comité de sélection a été constitué} gegeben. \\
Wie davor gehen wir monoton von links nach rechts und suchen nach möglichen Phrasen:

\vspace{7pt}
1. Phrase \glqq un\grqq\ besteht aus ein Wort, ist in TM $\Rightarrow$ übersetze die Phrase und shau nach den Phrasen danach.\\
Erste Phrase nach \glqq un\grqq\  ist  \glqq Comité\grqq\  $\Rightarrow$  vertausche die Phrasen und übersetze jede einzige vertauschte Phrase separat.\\
Zweite Phrase nach \glqq un\grqq\  ist  \glqq Comité de\grqq\  $\Rightarrow$  vertausche die Phrasenund übersetze jede einzige vertauschte Phrase separat.

\vspace{7pt}
2. Phrase \glqq un Comité\grqq\ besteht aus zwei Wörtern, ist in TM $\Rightarrow$ übersetze und shau nach den Phrasen danach.\\
Erste Phrase danach ist \glqq de\grqq\  $\Rightarrow$  vertausche die Phrasen und übersetze jede einzige vertauschte Phrase separat. \\
usw.

\vspace{20pt}
For Implementation siehe Datei {\bf decodeMy}. Hier ist ein Ausschnitt davon mit den meisten Änderungen.

Um nach einer gefundenen Phrase alle mögliche Phrasen danach zu finden, brauchen wir eine zusätzliche Schleife (Zeile $18$).
Die Schleife ueber $swap=1,2$ unterscheidet ob es sich um eine einzige Phrase oder um Vertauschung von zwei Phrasen handelt (Zeilen $20$, $23$ und $56$).\\
Wenn wir zwei nacheinandre folgende Phrasen $P_1$ und $P_2$ gefunden haben, dann werden in entsprechenden Stack alle Kombinationen von  möglichen Übersetzungen der Phrase $P_2$ gefolgt von allen möglichen Übersetzungen der Phrase $P_1$ (Zeile $74-115$).

\lstinputlisting[language=Python]{decodeMyCut}	 

\vspace{20pt}
Die Ergebnisse dieses Decoder sind :

\vspace{5pt}
$-1409.200122$ für Stackgröße $1$  (Basis-Decoder $-1442.727775$)\\
$-1391.747648$ für Stackgröße $1000$ (Basis-Decoder $-1439.213923$)\\

Bei dem Versuch die Pruning Konstante $s$ hoehe zu wählen, hat sich das Ergebnis nicht verbessert.
$-1391.747648$ für Stackgröße $5000$ (Basis-Decoder $-1439.213923$).
\end{document}
