\documentclass[
		a4paper,
		12pt,
		twoside=false,
		chapterprefix,
		numbers=noenddot,
		plainheadsepline
]{scrreprt} %{article}

\usepackage[utf8]{inputenc}		% UTF8-Kodierung für Umlaute usw
\usepackage[T1]{fontenc} 		% Ligaturen, richtige Umlaute im PDF 
\usepackage[german,ngerman]{babel} % Silbentrennung

%\usepackage{fancyhdr}
\usepackage{setspace} % Zeilenabstand
	\onehalfspacing % 1,5 Zeilen

% Graphik
\usepackage{graphicx}
\usepackage{wrapfig}
\usepackage{tikz}
	\usetikzlibrary[topaths,arrows,calc]
\usepackage{rotating}
% Bildunterschrift
\setcapindent{0em} % kein Einrücken der Caption von Figures und Tabellen
%\setcapwidth[c]{0.5\textwidth}

\usepackage{bm}
\setlength{\headheight}{15pt}

\usepackage{hyperref}
\setlength{\parindent}{0em}

\usepackage[paper=a4paper,left=30mm,right=30mm,top=20mm,bottom=20mm]{geometry}
\setcounter{tocdepth}{3}
\setcounter{secnumdepth}{3}

\usepackage[square,sort]{natbib}
% Mathe
\usepackage{amssymb,amsmath}
% Schrift-art
\usepackage{mathptmx}% Times Roman font

% Absätze
\setlength{\parindent}{0pt}
% Floatbarrier
% [verbose] 
% Schreibt viele zusätzliche Informationen in die Log-Datei.
\usepackage[verbose]{placeins}

% Schriften-Größen
\setkomafont{chapter}{\LARGE\rmfamily} % Überschrift der Ebene
\setkomafont{section}{\Large\rmfamily}
\setkomafont{subsection}{\large\rmfamily}
\setkomafont{subsubsection}{\large\rmfamily}
\setkomafont{chapterentry}{\large\rmfamily} % Überschrift der Ebene in Inhaltsverzeichnis
\setkomafont{descriptionlabel}{\bfseries\rmfamily} % für description Umgebungen
\setkomafont{captionlabel}{\small\bfseries}
%\setkomafont{caption}{\small}
\usepackage[font=small,labelfont=bf,labelsep=colon]{caption}
