\chapter{Zusammenfassung}

Der Schwerpunkt dieser Arbeit war Constraint Programmierung und ihre Anwendung in der Scheduling-Theorie.

Im ersten Teil wurden wichtige Begriffe und Algorithmen dieses Programmierparadigmas erklärt. Außerdem wurde auf die Vorteile des Constraint Programmierung eingegangen und ein kleiner Vergleich zwischen ihr und ganzzahliger Optimierung gemacht. Dies macht diese Arbeit zu einer kurzen Einführung in Constraint Programmierung für Anfänger. 

Im zweiten Teil wurden einige Beispiele aus der Scheduling Theorie betrachtet und wie sie mit Hilfe von Constraint Programmierung formuliert und gelöst werden können. Es wurde mit einem einfacheren Beispiel der Minimierung der Gesamtverspätung angefangen, um zu zeigen wie einfach ein Problem in CP formuliert werden kann. Und das Beispiel mit Timetabling am Ende zeigt noch mal, das Benutzung von Contraint Programmierung anstatt ganzzahliger Optimierung zu signifikanter Verbesserung der Laufzeit führen kann.

Es existieren aber weitere interessante Ansätze der CP im Bereich der Scheduling-Theorie, die in dieser Arbeit nicht betrachtet wurden. In dieser Richtung kann die Arbeit erweitert und vervollständigt werden. 